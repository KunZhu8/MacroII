\documentclass[12pt,notitlepage]{article}%
\usepackage[top=1.5in, bottom=1.5in, left=1.5in, right=1.5in]{geometry}
\usepackage{amsthm}
\usepackage{setspace}
\usepackage{eurosym}
\usepackage{subfigure}
\usepackage{xcolor}
\usepackage{url}
\usepackage{supertabular}
\usepackage{amssymb}
\usepackage{graphicx}
\usepackage[colorlinks,linkcolor=links,citecolor=cites,urlcolor=MyDarkBlue]%
{hyperref}
\usepackage{amsfonts}
\usepackage{amsmath}
\usepackage{amstext}
\usepackage{appendix}
\usepackage{tikz}
\usepackage{rotating}
\usepackage{lscape}
\usepackage{chngpage}
\usepackage{esint}
\usepackage{natbib}
\usepackage{bm}
\setcounter{MaxMatrixCols}{30}
\providecommand{\U}[1]{\protect\rule{.1in}{.1in}}
\newtheorem{theorem}{Theorem}
\newtheorem{acknowledgement}{Acknowledgement}
\newtheorem{algorithm}{Algorithm}
\newtheorem{Assumption}{Assumption}
\newtheorem{axiom}{Axiom}
\newtheorem{case}{Case}
\newtheorem{claim}{Claim}
\newtheorem{conclusion}{Conclusion}
\newtheorem{condition}{Condition}
\newtheorem{conjecture}{Conjecture}
\newtheorem{corollary}{Corollary}
\newtheorem{criterion}{Criterion}
\newtheorem{definition}{Definition}
\newtheorem{example}{Example}
\newtheorem{exercise}{Exercise}
\newtheorem{lemma}{Lemma}
\newtheorem{notation}{Notation}
\newtheorem{problem}{Problem}
\newtheorem*{assumption}{Asumption}
\newtheorem{proposition}{Proposition}
\newtheorem{remark}{Remark}
\newtheorem{solution}{Solution}
\newtheorem{summary}{Summary}
\newtheorem{observation}{Observation}
\numberwithin{equation}{section}
\DeclareMathOperator*{\argmax}{argmax}
\DeclareMathOperator*{\Prob}{Prob}
\setlength{\topmargin}{0in}
\setlength{\textheight}{8.8in}
\setlength{\oddsidemargin}{0.1in}
\setlength{\evensidemargin}{0.1in}
\setlength{\textwidth}{6.5in}
\setlength{\headheight}{0in}
\def \definitionname{Definition}
\def \sectionautorefname{Section}
\def \subsectionautorefname{Section}
\def \footnotename{footnote}
\def \examplename{Example}
\def \lemmaname{Lemma}
\def \propositionname{Proposition}
\def \appendixname{Appendix}
\def \assumptionname{Assumption}
\def \corollaryname{Corollary}
\def \remarkname{Remark}
\def \remname{Remark}
\providecommand{\possessivecite}[1]{\citeauthor{#1}'s\nolinebreak[2]
	(\citeyear{#1})}
\definecolor{MyDarkBlue}{rgb}{0,0.08,0.45}
\definecolor{cites}{HTML}{324b13}
\definecolor{links}{HTML}{1a663b}
\definecolor{MyLightMbuyera}{cmyk}{0.1,0.8,0,0.1}
\hypersetup{
	colorlinks,citecolor=blue,filecolor=black,linkcolor=blue,urlcolor=blue
}
\parindent= 0.6cm
\linespread{1.3}
\begin{document}
	
	\title{Macro Problem Set 1}
	\author{Kun Zhu}
	
	\maketitle
	
	\section{Neo-Classical Model}
		\subsection{Competitive equilibrium}
			\indent Arrow-Debreu competitive equilibrium in Neo-Classical model consists of prices $\{p_t,w_t,r_t\}_{t=0}^{\infty}$, allocations for the firm $\{y_t,k_t^d,l_t^d\}_{t=0}^{\infty}$ and the allocations for household $\{c_t,i_t,x_{t+1},k_t^s,l_t^s\}_{t=0}^{\infty}$ such that,
			\begin{itemize}
			\item[(i)] Given a sequence of prices $\{p_t,w_t,r_t\}_{t=0}^{\infty}$, the firm allocation $\{y_t,k_t^d,l_t^d\}_{t=0}^{\infty}$ solves the firm problem,
			\begin{equation}
			\begin{split}
				\Pi = \max_{\{y_t,k_t,l_t\}_{t=0}^{\infty}}&\sum_{t=0}^{\infty}
				p_t(y_t-r_tk_t-w_tl_t)\\
				\text{s.t.  }&y_t=f(k_t,l_t), \forall t\geq 0;\\
				&y_t,k_t,l_t\geq 0, \forall t \geq 0.
			\end{split}
			\end{equation}
			
			\item[(ii)] Given a sequence of prices $\{p_t,w_t,r_t\}_{t=0}^{\infty}$ and the profit of firm $\Pi$, the household allocation $\{c_t,i_t,x_{t+1},k_t^s,l_t^s\}_{t=0}^{\infty}$ solves the household problem,
			\begin{equation}
			\begin{split}
			\max_{\{c_t,i_t,x_{t+1},k_t,l_t\}_{t=0}^{\infty}}&\sum_{t=0}^{\infty}
			\beta ^tu(c_t)\\
			\text{s.t.  }& \sum_{t=0}^{\infty}p_t(c_t+i_t)
			\leq\sum_{t=0}^{\infty}p_t(r_tk_t+w_tl_t)+\Pi;\\
			&x_{t+1}=i_t, \forall t\geq 0;\\
			&0\leq l_t\leq 1,0\leq k_t\leq x_t, c_t\geq 0, x_{t+1}\geq 0,\forall t \geq 0;\\
			&x_0 \text{ is given.}
			\end{split}
			\end{equation}
			
			\item[(iii)] The market clear conditions,
			\begin{equation*}
			\begin{split}
				y_t&=c_t+i_t	\text{	(goods market)};\\
				l_t^d&=l_t^s	\text{	(labour market)};\\
				k_t^d&=k_t^s 	\text{	(capital market)}.
			\end{split}
			\end{equation*}
			\end{itemize}

		\subsection{Social planner problem}
			The problem of the social planner is that, given the initial capital $k_0$,
			\begin{equation}\label{SPP1}
				\begin{split}
				w( k_0)&=\max_{\{c_t, k_t, l_t \}_{t=0}^{\infty}}
				\sum_{t=0}^{\infty}\beta^tu(c_t)\\
				s.t. \;\;f(k_t,l_t)&=c_t+k_{t+1}, \;\;\forall t\geq 0\\
				c_t&\geq0,\;k_t\geq0,\;0\leq l_t\leq 1,  \;\;\forall t\geq 0\\
				k_0&\text{ is given.}
				\end{split}
			\end{equation}
		
		\subsection{Welfare theorem}
			\textbf{SPP}
			
			Assuming the standard property of utility function $u(\cdot)$ and production function $f(\cdot)$, we can see that in the solution of SPP, $l_t=1$. Thus the SPP can be rewritten as
			\begin{equation}\label{SPP2}
			\begin{split}
			w( k_0)&=\max_{\{ k_{t+1} \}_{t=0}^{\infty}}
			\sum_{t=0}^{\infty}\beta^tu(f(k_t,1)-k_{t+1})\\
			&0\leq k_{t+1}\leq f(k_t,1),  \;\;\forall t\geq 0,\\
			&k_0 \;is\; given. 
			\end{split}
			\end{equation}
			
			FOC of  for SPP  gives the Euler equation
			\begin{equation}
			u'(f(k_t,1)-k_{t+1})=\beta u'(f(k_t,1)-k_{t+1})f_k(k_t,1). 
			\end{equation}
			
			The transversality condition (hereafter, TVC) for SPP is
			\begin{equation}\label{EularSPP}
			\lim_{t\to \infty}\beta^t u'(f(k_t)-k_{t+1})f_k(k_t,1)k_t=0,
			\end{equation}
			or, equivalently, 
			\begin{equation}\label{TVCspp}
			\lim_{t\to \infty}\ \lambda_tk_{t+1}=0,
			\end{equation}
			where  $\lambda_t$ is the Lagrange multiplier for time $t$ in the  SPP \ref{SPP2}. 
			
			\textbf{ADCE}
			
			FOC for firm's problem, yields
			\begin{equation*}
			\begin{split}
					r_t&=f_k(k_t,1)\\
					w_t&=f_l(k_t,1)
			\end{split}
			\end{equation*}
			
			FOC for household's problem yields
			
			\begin{equation*}
			\begin{split}
			\beta^tu'(c_t)&=\mu p_t\\
			\beta^{t+1}u'(c_{t+1})&=\mu p_t r_{t+1}
			\end{split}
			\end{equation*}
			
			With market clearing 
			\begin{equation*}
			c_t=f(k_t)-k_{t+1},
			\end{equation*}
			we have the Euler equation for ADCE
			
			\begin{equation}\label{EularADCE}
			\lim_{t\to \infty}\beta^t u'(f(k_t)-k_{t+1})f'(k_t)k_t=0.
			\end{equation}
			
			TVC for household problem is given by 
			\begin{equation}\label{TVCADCE}
			\begin{split}
			\lim_{t\to \infty}\ p_tk_{t+1}&=\frac{1}{\mu}\lim_{t\to \infty}\beta^tu'(c_t)k_{t+1}\\
			&=\frac{1}{\mu}\lim_{t\to \infty}\beta^{t-1}u'(c_{t-1})k_{t}\\
			&=\frac{1}{\mu}\lim_{t\to \infty}\beta^{t-1}\beta u'(c_{t-1})r_tk_{t}\\
			&=\frac{1}{\mu}\lim_{t\to \infty}\beta^{t-1}\beta u'(f(k_t)-k_{t+1})k_{t}.
			\end{split}
			\end{equation}
			\textbf{Equivalence}
			
			With $p_t=\lambda_t$, the optimal allocation in  SPP and that in ADCE has the same Euler equation and TVCs, then the optimal in two problem are the same. Hence the desired result is obtained.

		\subsection{Social planner Dynamic programming problem}
			Let $l_t=1, c_t=f(k_t,l_t)-k_{t+1}+(1-\delta)k_t$, we can get the dynamic programming problem,
			\begin{equation*}
			\begin{split}
				\max_{\{k_t\}_{t=0}^{\infty}}\sum_{t=0}^{\infty}&\beta^tu[f(k_t,1)-k_{t+1}]\\
				s.t.\;\;&0\leq k_{t+1}\leq f(k_t,1);\\
				&k_0 \text{ is given.}
			\end{split}
			\end{equation*}
			Define the value function $V(k)$ as the value of the lifetime social planner problem given the initial capital as $k$, then we can get the Bellman equation,
			\begin{equation*}
				V(k)=\max_{0\leq k'\leq f(k,1)} \{u(f(k,1)-k')+\beta V(k')\}
			\end{equation*}
			
		\subsection{An example}
			Let $u(c)=log(c), f(k,l)=zk^{\alpha}l^{1-\alpha}$, the social planner dynamic programming problem becomes,
			\begin{equation*}
				V(k)=\max_{0\leq k'\leq zk^{\alpha}} \{log(zk^{\alpha}-k')+\beta V(k')\}
			\end{equation*}
			Guess a solution for $V$ is that $V(k)=mlog(k)+n$, then the $FOC$ for the problem is $$-\frac{1}{zk^{\alpha}-k'}+\frac{\beta m}{k'}=0,$$ thus $k'=\frac{\beta mzk^{\alpha}}{1+\beta m}$, then we can solve $m$ and $n$,
			\begin{equation*}
			\begin{split}
				m=&\frac{\alpha}{1-\alpha \beta},\\
				n=&\frac{1}{(1-\beta)(1-\alpha \beta)}[(\alpha \beta)log(\alpha \beta)+(1-\alpha \beta)log(1-\alpha \beta)+log(z)],
			\end{split}
			\end{equation*}
			Then the policy function is $g(k)=k'|_{m=\frac{\alpha}{1-\alpha \beta}}=\alpha \beta zk^{\alpha}$
		
		\subsection{Steady state}
			For steady state, let $g(k)=k$ to get $k_s$, then
			\begin{equation*}
			\begin{split}
				k_s&=(\alpha \beta z)^{\frac{1}{1-\alpha}};\\
				c_s&=f(k_s,1)-g(k_s)=zk_s^{\alpha}-\alpha \beta zk_s^{\alpha}=(1-\alpha \beta)z(\alpha \beta z)^{\frac{\alpha}{1-\alpha}};\\
				r_s&=f_k(k_s,1)=\alpha zk_s^{\alpha -1}=\frac{1}{\beta};\\
				w_s&=f_l(k_s,1)=(1-\alpha) zk_s^{\alpha}=(1-\alpha) z(\alpha \beta z)^{\frac{\alpha}{1-\alpha}};\\
				y_s&=f(k_s,1)=zk_s^{\alpha}=z(\alpha \beta z)^{\frac{\alpha}{1-\alpha}}.
			\end{split}
			\end{equation*} 
		
		\subsection{Characterization for unsteady growth path}
			For arbitrary initial $k_0$,
			\begin{equation*}
			\begin{split}
				k_{t+1}&=g(k_t)=\alpha \beta z k_t^{\alpha};\\
				c_t&=f(k_t,1)-g(k_t)=zk_t^{\alpha}-\alpha \beta zk_t^{\alpha}=(1-\alpha \beta)zk_t^{\alpha};\\
				r_t&=f_k(k_t,1)=\alpha zk_t^{\alpha -1};\\
				w_t&=f_l(k_t,1)=(1-\alpha) zk_t^{\alpha};\\
				y_t&=f(k_t,1)=zk_t^{\alpha}.
			\end{split}
			\end{equation*} 
	\setcounter{table}{0}
	
	\bigskip

\end{document}

comm

