\documentclass[12pt,notitlepage]{article}%
\usepackage[top=1.5in, bottom=1.5in, left=1.5in, right=1.5in]{geometry}
\usepackage{amsthm}
\usepackage{setspace}
\usepackage{eurosym}
\usepackage{subfigure}
\usepackage{xcolor}
\usepackage{url}
\usepackage{supertabular}
\usepackage{amssymb}
\usepackage{graphicx}
\usepackage[colorlinks,linkcolor=links,citecolor=cites,urlcolor=MyDarkBlue]%
{hyperref}
\usepackage{amsfonts}
\usepackage{amsmath}
\usepackage{amstext}
\usepackage{appendix}
\usepackage{tikz}
\usepackage{rotating}
\usepackage{lscape}
\usepackage{chngpage}
\usepackage{esint}
\usepackage{natbib}
\usepackage{bm}
\setcounter{MaxMatrixCols}{30}
\providecommand{\U}[1]{\protect\rule{.1in}{.1in}}
\newtheorem{theorem}{Theorem}
\newtheorem{acknowledgement}{Acknowledgement}
\newtheorem{algorithm}{Algorithm}
\newtheorem{Assumption}{Assumption}
\newtheorem{axiom}{Axiom}
\newtheorem{case}{Case}
\newtheorem{claim}{Claim}
\newtheorem{conclusion}{Conclusion}
\newtheorem{condition}{Condition}
\newtheorem{conjecture}{Conjecture}
\newtheorem{corollary}{Corollary}
\newtheorem{criterion}{Criterion}
\newtheorem{definition}{Definition}
\newtheorem{example}{Example}
\newtheorem{exercise}{Exercise}
\newtheorem{lemma}{Lemma}
\newtheorem{notation}{Notation}
\newtheorem{problem}{Problem}
\newtheorem*{assumption}{Asumption}
\newtheorem{proposition}{Proposition}
\newtheorem{remark}{Remark}
\newtheorem{solution}{Solution}
\newtheorem{summary}{Summary}
\newtheorem{observation}{Observation}
\numberwithin{equation}{section}
\DeclareMathOperator*{\argmax}{argmax}
\DeclareMathOperator*{\Prob}{Prob}
\setlength{\topmargin}{0in}
\setlength{\textheight}{8.8in}
\setlength{\oddsidemargin}{0.1in}
\setlength{\evensidemargin}{0.1in}
\setlength{\textwidth}{6.5in}
\setlength{\headheight}{0in}
\def \definitionname{Definition}
\def \sectionautorefname{Section}
\def \subsectionautorefname{Section}
\def \footnotename{footnote}
\def \examplename{Example}
\def \lemmaname{Lemma}
\def \propositionname{Proposition}
\def \appendixname{Appendix}
\def \assumptionname{Assumption}
\def \corollaryname{Corollary}
\def \remarkname{Remark}
\def \remname{Remark}
\providecommand{\possessivecite}[1]{\citeauthor{#1}'s\nolinebreak[2]
	(\citeyear{#1})}
\definecolor{MyDarkBlue}{rgb}{0,0.08,0.45}
\definecolor{cites}{HTML}{324b13}
\definecolor{links}{HTML}{1a663b}
\definecolor{MyLightMbuyera}{cmyk}{0.1,0.8,0,0.1}
\hypersetup{
	colorlinks,citecolor=blue,filecolor=black,linkcolor=blue,urlcolor=blue
}
\parindent= 0.6cm
\linespread{1.3}
\begin{document}
	
	\title{Macro Problem Set 1}
	\author{Kun Zhu}
	
	\maketitle
	
	\section{Neo-Classical Model}
		\subsection{Competitive equilibrium}
			\indent Arrow-Debreu competitive equilibrium in Neo-Classical model consists of prices $\{p_t,w_t,r_t\}_{t=0}^{\infty}$, allocations for the firm $\{y_t,k_t^d,l_t^d\}_{t=0}^{\infty}$ and the allocations for household $\{c_t,i_t,x_{t+1},k_t^s,l_t^s\}_{t=0}^{\infty}$ such that,
			\begin{itemize}
			\item[(i)] Given a sequence of prices $\{p_t,w_t,r_t\}_{t=0}^{\infty}$, the firm allocation $\{y_t,k_t^d,l_t^d\}_{t=0}^{\infty}$ solves the firm problem,
			\begin{equation}
			\begin{split}
				\Pi = \max_{\{y_t,k_t,l_t\}_{t=0}^{\infty}}&\sum_{t=0}^{\infty}
				p_t(y_t-r_tk_t-w_tl_t)\\
				\text{s.t.  }&y_t=f(k_t,l_t)=zk_t^{\alpha}l_t^{1-\alpha}, \forall t\geq 0;\\
				&y_t,k_t,l_t\geq 0, \forall t \geq 0.
			\end{split}
			\end{equation}
			
			\item[(ii)] Given a sequence of prices $\{p_t,w_t,r_t\}_{t=0}^{\infty}$ and the profit of firm $\Pi$, the household allocation $\{c_t,i_t,x_{t+1},k_t^s,l_t^s\}_{t=0}^{\infty}$ solves the household problem,
			\begin{equation}
			\begin{split}
				\max_{\{c_t,i_t,x_{t+1},k_t,l_t\}_{t=0}^{\infty}}&\sum_{t=0}^{\infty}
				\beta ^tu(c_t,l_t)=\sum_{t=0}^{\infty}
				\beta ^t(\frac{c_t^{1-\sigma}}{1-\sigma}-\chi \frac{l_t^{1+\eta}}{1+\eta})\\
				\text{s.t.  }& \sum_{t=0}^{\infty}p_t(c_t+i_t)
				\leq\sum_{t=0}^{\infty}p_t(r_tk_t+w_tl_t)+\Pi;\\
				&x_{t+1}=i_t+(1-\delta)x_t, \forall t\geq 0;\\
				&0\leq l_t\leq 1,0\leq k_t\leq x_t, c_t\geq 0, x_{t+1}\geq 0,\forall t \geq 0;\\
				&x_0 \text{ is given.}
			\end{split}
			\end{equation}
			
			\item[(iii)] The market clear conditions,
			\begin{equation*}
			\begin{split}
				y_t&=c_t+i_t	\text{	(goods market)};\\
				l_t^d&=l_t^s	\text{	(labour market)};\\
				k_t^d&=k_t^s 	\text{	(capital market)}.
			\end{split}
			\end{equation*}
			\end{itemize}
		
		\subsection{Steady state}
			For firm problem,
			\begin{equation}
			\begin{split}
				r_t =&z\alpha k_t^{\alpha -1}l_t^{1-\alpha},\\
				w_t =&z(1-\alpha) k_t^{\alpha}l_t^{-\alpha}.\\
				y_t=&r_tk_t+w_tl_t=zk_t^{\alpha}l_t^{1-\alpha}\\
				\Pi =&0.
			\end{split}
			\end{equation}
			
			Then for household problem, obviously, it is optimal to choose $k_t=x_t$, combining the production function and goods market clear condition. We can write the Lagrange function and FOC for that(assuming interior solution),
			\begin{equation}
			\begin{split}
				\mathcal{L}(\{c_t,k_{t+1},l_t\}_{t=0}^{\infty};\lambda)=&\sum_{t=0}^{\infty}
				\beta ^t(\frac{c_t^{1-\sigma}}{1-\sigma}-\chi \frac{l_t^{1+\eta}}{1+\eta})+\lambda(\sum_{t=0}^{\infty}p_t(zk_t^{\alpha}l_t^{1-\alpha}-c_t-k_{t+1}+(1-\delta)k_t)\\
				\frac{\partial \mathcal{L}}{\partial c_t}=&\beta^tc_t^{-\sigma}-\lambda p_t=0,\\
				\frac{\partial \mathcal{L}}{\partial l_t}=&-\beta^t\chi l_t^{\eta}+\lambda p_tz(1-\alpha) k_t^{\alpha}l_t^{-\alpha}=0,\\
				\frac{\partial \mathcal{L}}{\partial k_t}=&p_t(z\alpha k_t^{\alpha -1}l_t^{1-\alpha}+1-\delta)-p_{t-1}=0,\\
				c_t=&zk_t^{\alpha}l_t^{1-\alpha}-k_{t+1}+(1-\delta)k_t(\text{use goods market clear to replace}\frac{\partial \mathcal{L}}{\partial \lambda}).
			\end{split}
			\end{equation}
			For Steady state, we can normalize $p_0=1$, $$\frac{\partial \mathcal{L}}{\partial c_t}=0 \Rightarrow p_t=\beta^t\Rightarrow \lambda =c_t^{-\sigma},$$ 
			then $$\frac{\partial \mathcal{L}}{\partial l_t}=0,\frac{\partial \mathcal{L}}{\partial k_t}=0,\frac{\partial \mathcal{L}}{\partial \lambda}=0  \Rightarrow$$
			\begin{equation}\label{ssfoc}
			\begin{split}
				&c^{\sigma} l^{\eta}=
				z(1-\alpha)k^{\alpha}l^{-\alpha}/\chi\\
				&z\alpha k^{\alpha -1}l^{1-\alpha}=
				1/\beta -1 +\delta\\
				&c=zk^{\alpha}l^{1-\alpha}-\delta k
			\end{split}
			\end{equation}	
			Use second one in \ref{ssfoc}, we have 
			\begin{equation*}
				M=\frac{k}{l}=\left(\frac{z\alpha \beta}{1-\beta +\beta \delta}\right)^{\frac{1}{1-\alpha}},
			\end{equation*}
			Plug in third one in \ref{ssfoc},
			\begin{equation*}
				N=\frac{c}{l}=z(\frac{k}{l})^{\alpha}-\delta \frac{k}{l}=zM^{\alpha}-\delta M,
			\end{equation*}
			Use $c=lN$ in first one in \ref{ssfoc},
			\begin{equation*}
			\begin{split}
				(lN)^{\sigma}l^{\eta}=&
				z(1-\alpha)(\frac{k}{l})^{\alpha}/\chi\Rightarrow\\
				l^{\sigma+\eta}&=\frac{z(1-\alpha)M^{\alpha}}{\chi N^{\sigma}} \Rightarrow\\
				l=&\left(\frac{z(1-\alpha)M^{\alpha}}{\chi N^{\sigma}}\right)^{\frac{1}{\sigma+\eta}},
			\end{split}
			\end{equation*}
			Then we can have all steady state variable,
			\begin{equation*}
			\begin{split}
				k=&Ml\\
				c=&Nl\\
				y=&zk^{\alpha}l^{1-\alpha}=zM^{\alpha}l\\
				r=&\alpha zk^{\alpha-1}l^{1-\alpha}=\alpha zM^{\alpha -1}\\
				w=&(1-\alpha)zk^{\alpha}l^{\alpha}=(1-\alpha)zM^{\alpha}.
			\end{split}
			\end{equation*}
		\subsection{Social planner problem}
			The problem of the social planner is that, given the initial capital $k_0$,
			\begin{equation}\label{SPP1}
				\begin{split}
					w( k_0)&=\max_{\{c_t, k_t, l_t \}_{t=0}^{\infty}}
					\sum_{t=0}^{\infty}
					\beta ^t(\frac{c_t^{1-\sigma}}{1-\sigma}-\chi \frac{l_t^{1+\eta}}{1+\eta})\\
					s.t. \;\;zk_t^{\alpha}l_t^{1-\alpha}&=c_t+k_{t+1}-(1-\delta)k_t, \;\;\forall t\geq 0\\
					c_t&\geq0,\;k_t\geq0,\;0\leq l_t\leq 1,  \;\;\forall t\geq 0\\
					k_0&\text{ is given.}
				\end{split}
			\end{equation}
			Bellman equation,
			\begin{equation}
				V(k)=\max_{\begin{smallmatrix}0\leq l\leq 1
					\\0\leq k'\leq zk^{\alpha}l^{1-\alpha}+(1-\delta)k\end{smallmatrix}}
					\{\frac{(zk^{\alpha}l^{1-\alpha}+(1-\delta)k-k')^{1-\sigma}}{1-\sigma}-\chi \frac{l^{1+\eta}}{1+\eta}+\beta \mathbb{E}V(k')\}
			\end{equation}

	\setcounter{table}{0}
	
	\bigskip

\end{document}

comm

